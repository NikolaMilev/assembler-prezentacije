\documentclass[a4paper, 10pt]{article}
\usepackage{titlesec}
\usepackage[margin=1.2cm]{geometry}
\usepackage[utf8]{inputenc}
\usepackage[T1]{fontenc}
\usepackage{fancyhdr}
\usepackage[ddmmyyyy]{datetime}
\begin{document}

\title{Asemblersko programiranje za x86\_64 arhitekturu}
\author{Nikola Milev}
\date{Poslednja izmena: \today}
\maketitle

\section[1]{Sintaksa}
	Na kursu ćemo koristiti Intelovu neprefiksnu sintaksu. \\ 
	Svaki red asemblera može biti:
	\begin{itemize}
		\item Prazan red: prazni redovi se ignorišu
		\item Direktiva
		\item Linije koje nisu ni prazne ni direktive smatraju se instrukcijama
	\end{itemize}
Pri nailasku na simbol $\#$, ostatak linije se ignoriše (komentari). Svaka linija može početi labelom.
	\subsection{Labele}
Definicija labele sastoji se iz identifikatora iza kog se navodi simbol $:$. Identifikator mora početi slovom ili simbolom $\_$, dok može sadržati slova, simbol $\_$ i cifre. Labele se prilikom prevođenja programa prevode u memorijske adrese. Labele mogu označavati adrese podataka, kao i adrese instrukcija.
	\subsection{Direktive}
	Direktive počinju simbolom $.$ i imaju specijalno značenje. 
	\begin{itemize}
		\item $.intel\_syntax$ $noprefix$ -- Označava se da se koristi Intelova neprefiksna sintaksa
		\item $.globl$ $identifikator$ ili $.global$ $identifikator$ -- Navodi se da je $identifikator$ globalni simbol
		\item $.data$ -- Počinje se sekcija inicijalizovanih podataka
		\item $.text$ -- Počinje se sekcija koda
		\item $.asciz$ -- Kreira se ASCI niska na čijem se kraju automatski navodi terminirajuća nula
		\item $.byte$ -- Kreira se jedan ili niz bajtova; članovi niza razdvojeni su zapetom
		\item $.word$ -- Kreira se jedan ili niz slogova od 2 bajta
		\item $.long$ -- Kreira se jedan ili niz slogova od 4 bajta
		\item $.quad$ -- Kreira se jedan ili niz slogova od 8 bajtova
	\end{itemize}
	\subsection{Instrukcije}
	Instrukcija se sastoji od koda instrukcije i operan(a)da. Svaki kod instrukcije ima svoju simboličku oznaku. Opšti oblik instrukcije sa dva operanda je: $kod$ $op1, op2$.
Načini zadavanja operanada:	
	\begin{itemize}
		\item Registarski operandi: navodi se simbolička oznaka registra
		\item Neposredni operandi: direktno se navodi vrednost sa kojom se radi
		\item Memorijski operandi: navodi se adresa sa kojom se radi. Opšti sintaksni oblik je: $[B + S*I + D]$. $B$ je bazna adresa, $D$ je pomeraj, $I$ je indeks, dok je $S$ veličina "elementa". Svaki od navedenih elemenata može se izostaviti i tada se dobijaju specijalni slučajevi:
		\begin{itemize}
			\item $[B]$ -- Bazno adresiranje
			\item $[B + D]$ -- Bazno adresiranje sa pomerajem
			\item $[B + S*I]$ -- Indeksno adresiranje
		\end{itemize}
	\end{itemize}
\section{Registri} 
U toku kursa, koristiće se registri opšte namene, veličine 8 bajtova (64 bita). Njih ima 16:
	\begin{itemize}
		\item $rax$, $rbx$, $rcx$, $rdx$, $rsi$, $rdi$, $rsp$, $rbp$, $r8$, ..., $r15$
	\end{itemize}
Moguć je pristup niža 4 bajta:
	\begin{itemize}
		\item $eax$, $ebx$, $ecx$, $edx$, $esi$, $edi$, $esp$, $ebp$, $r8d$, ..., $r15d$
	\end{itemize}
	Moguć je pristup najniža 2 bajta:
	\begin{itemize}
		\item $ax$, $bx$, $cx$, $dx$, $si$, $di$, $sp$, $bp$, $r8w$, ..., $r15w$
	\end{itemize}
	Moguć je pristup najnižem bajtu:
	\begin{itemize}
		\item $al$, $bl$, $cl$, $dl$, $sil$, $dil$, $spl$, $bpl$, $r8b$, ..., $r15b$
	\end{itemize}
Registri $rbp$ i $rsp$ imaju specijalnu svrhu pri radu sa stekom:
	\begin{itemize}
			\item Registar $rbp$ služi za čuvanje trenutnog okvira steka.
			\item Registar $rsp$ služi za čuvanje trenutnog vrha steka. Pri izvršavanju instrukcije $pop$, vrednost registra $rsp$ automatski se uvećava za veličinu operanda, dok se pri instrukciji $push$ vrednost registra $rsp$ automatski umanjuje za veličinu operanda.
	\end{itemize}
\section{Instrukcije}
	\subsection{Instrukcije za rad sa stekom}
		\begin{itemize}
			\item $push$ $op$ -- Na vrh steka smešta se vrednost sadržana u operandu. Registar $rsp$ umanjuje se za veličinu operanda. Na stek je moguće smestiti samo dvobajtne i osmobajtne podatke.
		 \item $pop$ $op$ -- Sa vrha steka uklanja se vrednost i smešta se u operand. registar $rsp$ uvećava se za veličinu operanda. Sa steka je moguće ukloniti samo dvobajtne i osmobajtne podatke. 
		 \end{itemize}
	\subsection{Instrukcije transfera}
		\begin{itemize}
			\item $mov$ $op1,$ $op2$ -- U prvi operand smešta se vrednost drugog operanda. Očekuje se da oba operanda budu iste širine.
			\item $movzx$ $op1$, $op2$ -- U prvi operand smešta se vrednost drugog operanda, proširena nulama (transfer sa neoznačenim proširivanjem). Drugi operand ne može biti dat neposredno. Očekivano je da je drugi operand manje širine od prvog.
			\item $movsx$ $op1$, $op2$ -- U prvi operand smešta se vrednost drugog operanda, proširena u skladu sa znakom (transfer sa označenim proširivanjem).  Drugi operand ne može biti dat neposredno. Očekivano je da je drugi operand manje širine od prvog.
			\item $lea$ $op1,$ $op2$ -- U drugi operand smešta se adresa koja je predstavljena drugim operandom.
		\end{itemize}
	\subsection{Aritmetičke instrukcije}
		% Pogledaj ovaj link kako bi razjasnio sve: https://en.wikibooks.org/wiki/X86_Assembly/Data_Transfer
		\begin{itemize}
			\item Sabiranje: $add$ $op1,$ $op2$ -- sabira argumente i rezultat smešta u prvi argument
			\item Oduzimanje: $sub$ $op1,$ $op2$ -- oduzima argumente i rezultat smešta u prvi argument
			\item Neoznačeno množenje: $mul$ $op$ -- Ukoliko je operand osmobajtni, neoznačeno množi sadržaj registra $rax$ operandom $op$ i rezultat se smešta u $rdx:rax$. Ukoliko je operand četvorobajtni, neoznačeno množi sadržaj registra $eax$ operandom $op$ i rezultat se smešta u $edx:eax$.
			\item Označeno množenje: $imul$ $op$ -- Ukoliko je operand osmobajtni, označeno množi sadržaj registra $rax$ operandom $op$ i rezultat se smešta u $rdx:rax$. Ukoliko je operand četvorobajtni, označeno množi sadržaj registra $eax$ operandom $op$ i rezultat se smešta u $edx:eax$.
			\item Proširivanje znaka: cdqe -- označeno proširivanje registra $eax$ na $rax$
			\item Proširivanje znaka: cdq -- označeno proširivanje registra $eax$ na $edx:eax$
			\item Proširivanje znaka: cqo -- označeno proširivanje registra $rax$ na $rdx:rax$
			\item Neoznačeno deljenje: $div$ $op$ --ukoliko je operand osmobajtni, neoznačeno deli sadržaj registara $rdx:rax$ operandom $op$; količnik se nalazi u $rax$ dok se ostatak pri deljenju nalazi u $rdx$ . Ukoliko je operand četvorobajtni, neoznačeno deli sadržaj registara $edx:eax$ operandom $op$; količnik se nalazi u $eax$ dok se ostatak pri deljenju nalazi u $edx$.
			\item Označeno deljenje: $idiv$ $op$ --ukoliko je operand osmobajtni, označeno deli sadržaj registara $rdx:rax$ operandom $op$; količnik se nalazi u $rax$ dok se ostatak pri deljenju nalazi u $rdx$ . Ukoliko je operand četvorobajtni, označeno deli sadržaj registara $edx:eax$ operandom $op$; količnik se nalazi u $eax$ dok se ostatak pri deljenju nalazi u $edx$.
			\item Negiranje: $neg$ $op$
			\item Uvećanje operanda za 1: $inc$ $op$
			\item Umanjenje operanda za 1: $dec$ $op$
		\end{itemize}
	\subsection{Logičke instrukcije}
	\begin{itemize}
		\item Konjunkcija: $and$ $op1$, $op2$
		\item Disjunkcija: $or$ $op1$, $op2$
		\item Ekskluzivna disjunkcija: $xor$ $op1$, $op2$
		\item Negacija: $not$ $op$
		\item Šiftovanje (pomeranje) u levo: $shl$ $op1$, $op2$ -- pomera prvi argument za broj mesta koji je sadržan u drugom argumentu. Drugi argument mora biti dat neposredno.
		\item Šiftovanje (pomeranje) u desno:
			\begin{itemize}
				\item Logičko: $shr$ $op1$, $op2$ ; Drugi argument mora biti dat neposredno.
				\item Aritmetičko: $sar$ $op1$, $op2$ ; Drugi argument mora biti dat neposredno.
			\end{itemize}
		
	\end{itemize}
	\subsection{Instrukcije poređenja}
	\begin{itemize}
		\item $cmp$ -- poređenje operanada oduzimanjem; ne menja operande nego $rflags$ registar
		\item $test$ -- bitovsko poređenje operanada konjukcijom; ne menja operande nego $rflags$ registar
	\end{itemize}
	\subsection{Instrukcije kontrole toka}
	\begin{itemize}
		\item Bezuslovni skok: $jmp$ $op$ -- bezuslovni skok na adresu
		\item Poziv funkcije: $call$ $op$ -- bezuslovni skok uz pamćenje povratne adrese na steku
		\item $ret$ -- skidanje adrese sa steka i skok na nju
		\item $jz$ $op$ -- Skok na adresu ukoliko je rezultat prethodne instrukcije nula
		\item $je$ $op$ -- Skok na adresu ukoliko je rezultat prethodnog poređenja jednako (ekvivalentno sa $jz$)
		\item $jnz$ $op$ -- Skok na adresu ukoliko rezultat prethodnog poređenja nije nula
		\item $jne$ $op$ -- Skok na adresu ukoliko rezultat prethodnog poređenja nije jednako (ekvivalentno sa $jnz$)
		\item $ja$ $op$ -- Skok ukoliko je rezultat prethodnog poređenja veće (neoznačeni brojevi)
		\item $jb$ $op$ -- Skok ukoliko je rezultat prethodnog poređenja manje (neoznačeni brojevi)
		\item $jae$ $op$ -- Skok ukoliko je rezultat prethodnog poređenja veće ili jednako (neoznačeni brojevi)
		\item $jbe$ $op$ -- Skok ukoliko je rezultat prethodnog poređenja manje ili jednako (neoznačeni brojevi)
		\item $jg$ $op$ -- Skok ukoliko je rezultat prethodnog poređenja veće (označeni brojevi)
		\item $jl$ $op$ -- Skok ukoliko je rezultat prethodnog poređenja manje (označeni brojevi)
		\item $jge$ $op$ -- Skok ukoliko je rezultat prethodnog poređenja veće ili jednako (označeni brojevi)
		\item $jle$ $op$ -- Skok ukoliko je rezultat prethodnog poređenja manje ili jednako (označeni brojevi)
		\item Postoje i negacije navedenih instrukcija: $jna$, $jnb$, $jnae$, $jnbe$, $jng$, $jnl$, $jnge$, $jnle$
	\end{itemize}
\subsection{Širina operanada u instrukciji}
Ukoliko kao jedan od operanada navedemo registar, njegova širina implicitno određuje širinu drugog operanda. Međutim, ukoliko nijedan operand nije registarski, neophodno je eksplicitno naglasiti veličinu bar jednog operanada:
	\begin{itemize}
		\item $byte$ $ptr$ -- Tretira operand kao jednobajtni podatak.
		\item $word$ $ptr$ -- Tretira operand kao dvobajtni podatak.
		\item $dword$ $ptr$ -- Tretira operand kao četvorobajtni podatak.
		\item $qword$ $ptr$ -- Tretira operand kao osmobajtni podatak.
	\end{itemize}
\section{Konvencije za pozivanje i pisanje funkcija}
\begin{itemize}
\item Poziv se vrši instrukcijom $call$ $op$
	\item Prenos parametara: Celobrojni parametri prenose se redom (s leva na desno) u registrima: $rdi$, $rsi$, $rdx$, $rcx$, $r8$, $r9$. Ukoliko funkciji prenosimo više od 6 argumenata, tada se preostali smeštaju na stek, redom s desna na levo.
	\item Povratna vrednost nalazi se u $rax$ registru.
	\item Registri koji pripadaju pozvanoj funkciji:
		\begin{itemize}
			\item $rax$, $rdi$, $rsi$, $rdx$, $rcx$, $r8-r11$
		\end{itemize}
	\item Registri koji pripadaju pozivajućoj funkciji:
		\begin{itemize}
			\item $rbx$, $rbp$, $rsp$, $r12-r15$
		\end{itemize}
	\item Ukoliko u našem programu pozivamo neku funkciju, neophodno je da u trenutku poziva adresa vrha steka (sadržana u registru $rsp$) bude deljiva sa 16.
	\item Pisanje prologa: instrukcija $enter$ $n$, $0$ gde je $n$ broj bajtova koji odvajamo za lokalne promenljive
	\item Pisanje epiloga: instrukcija $leave$, koja vraća stek u stanje u trenutku ulaska u funkciju.
	\item Povratak iz funkcije: instrukcija $ret$
\end{itemize}
	
% vidi ovo dj kako da prevedes lepo
\section{Prevođenje}
Za prevođenje asemblerskog koda dovoljno je koristiti $gcc$. Na primer, ukoliko prevodimo dva fajla, $1.c$ i $1.s$, tada bi naredba za prevođenje bila $gcc$ $1.c$ $1.s$.
\end{document}