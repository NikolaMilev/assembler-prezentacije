\documentclass[a4paper, 10pt]{article}
\usepackage{titlesec}
\usepackage[margin=1.2cm]{geometry}
\usepackage[utf8]{inputenc}
\usepackage[T1]{fontenc}
\usepackage{fancyhdr}
\usepackage[ddmmyyyy]{datetime}
\begin{document}

\title{Asemblersko programiranje za x86\_64 arhitekturu}
\author{Nikola Milev}
\date{Poslednja izmena: \today}
\maketitle

\section[1]{Sintaksa}
	Na kursu ćemo koristiti Intelovu neprefiksnu sintaksu. \\ 
	Svaki red asemblera može biti:\\
	\begin{itemize}
		\item Prazan red: prazni redovi se ignorišu
		\item Direktiva
		\item Komentar: Pri nailasku na simbol $\#$, ostatak linije se ignoriše
		\item Linije koje nisu ni prazne ni direktive smatraju se instrukcijama
		\item Svaka linija može početi labelom
	\end{itemize}
	\subsection{Labele}
Definicija labele sastoji se iz identifikatora iza kog se navodi simbol $:$. Identifikator mora početi slovom ili simbolom $\_$, dok može sadržati slova, simbol $\_$ i cifre. Labele se prilikom prevođenja programa prevode u memorijske adrese. Labele mogu označavati adrese podataka, kao i instrukcija.
	\subsection{Direktive}
	Direktive počinju simbolom $.$ i imaju specijalno značenje. 
	\begin{itemize}
		\item $.intel\_syntax$ $noprefix$ -- Označava se da se koristi Intelova neprefiksna sintaksa
		\item $.globl$ $identifikator$ ili $.global$ $identifikator$ -- Navodi se da je $identifikator$ globalni simbol
		\item $.data$ -- Počinje se sekcija inicijalizovanih podataka
		\item $.text$ -- Počinje se sekcija koda
		\item $.asciz$ -- Kreira se ASCI niska na čijem se kraju automatski navodi terminirajuća nula
		\item $.byte$ -- Kreira se jedan ili niz bajtova; članovi niza razdvojeni su zapetom
		\item $.word$ -- Kreira se jedan ili niz slogova od 2 bajta
		\item $.long$ -- Kreira se jedan ili niz slogova od 4 bajta
		\item $.quad$ -- Kreira se jedan ili niz slogova od 8 bajtova
	\end{itemize}
	\subsection{Instrukcije}
	Instrukcija se sastoji od koda operanda i operan(a)da. Svaki kod instrukcije ima svoju simboličku oznaku. Opšti oblik instrukcije sa dva operanda je: $kod$ $op1, op2$.
Načini zadavanja operanada:	
	\begin{itemize}
		\item Registarski operandi: navodi se simbolička oznaka registra
		\item Neposredni operandi: direktno se navodi vrednost sa kojom se radi
		\item Memorijski operandi: navodi se adresa na kojoj se nalazi vrednost sa kojom se radi. Opšti sintaksni oblik je: $[B + S*I + D]$. $B$ je bazna adresa, $D$ je pomeraj, $I$ je indeks, dok je $I$ veličina "elementa". Svaki od navedenih elemenata može se izostaviti i tada se dobijaju specijalni slučajevi:
		\begin{itemize}
			\item $[B]$ -- Bazno adresiranje
			\item $[B + D]$ -- Bazno adresiranje sa pomerajem
			\item $[B + S*I]$ -- Indeksno adresiranje
		\end{itemize}
	\end{itemize}
\section{Registri} 
\section{Instrukcije}
\section{Konvencije za pozivanje funkcija}
\section{Konvencije za pisanje funkcija} Mozda ova dva u jedan
\section{Prevođenje}

\end{document}